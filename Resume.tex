%-------------------------
% Resume in Latex
% Author : ETH juniors
% Inspired by: https://github.com/sb2nov/resume
% License : MIT
% Website : www.ethjuniors.ch
%------------------------

\documentclass[letterpaper,11pt]{article}

\usepackage{latexsym}
\usepackage[empty]{fullpage}
\usepackage{titlesec}
\usepackage{marvosym}
\usepackage[usenames,dvipsnames]{color}
\usepackage{verbatim}
\usepackage{enumitem}
\usepackage[hidelinks]{hyperref}
\usepackage{fancyhdr}
\usepackage[english]{babel}
\usepackage{tabularx}
\usepackage{bibentry}
%\usepackage{biblatex}
\input{glyphtounicode}



% Custom font
\usepackage[default]{lato}

\pagestyle{fancy}
\fancyhf{} % clear all header and footer fields
\fancyfoot{}
\renewcommand{\headrulewidth}{0pt}
\renewcommand{\footrulewidth}{0pt}

% Adjust margins
\addtolength{\oddsidemargin}{-0.5in}
\addtolength{\evensidemargin}{-0.5in}
\addtolength{\textwidth}{1in}
\addtolength{\topmargin}{-.5in}
\addtolength{\textheight}{1.0in}

\urlstyle{same}

\raggedbottom
\raggedright
\setlength{\tabcolsep}{0in}

% Sections formatting
\titleformat{\section}{
  \vspace{-4pt}\scshape\raggedright\large
}{}{0em}{}[\color{black}\titlerule\vspace{-5pt}]

% Ensure that generate pdf is machine readable/ATS parsable
\pdfgentounicode=1

% Package for listing publications in resume

\usepackage[style=ieee]{biblatex}
\addbibresource{publications.bib}
\DeclareFieldFormat{labelnumberwidth}{}
\setlength{\biblabelsep}{0pt}
\setlength{\biblabelsep}{0pt}
\DeclareBibliographyCategory{conf}
\DeclareBibliographyCategory{publ}
\addtocategory{conf}{moraux}
\addtocategory{publ}{angenendt,baez/article}



%-------------------------%
% Custom commands
\begin{document}
%-------------------------%
% Custom commands
\newcommand{\resumeItem}[1]{
  \item\small{
    {#1 \vspace{-2pt}}
  }
}

\newcommand{\resumeSubheading}[4]{
  \vspace{-2pt}\item
    \begin{tabular*}{0.97\textwidth}[t]{l@{\extracolsep{\fill}}r}
      \textbf{#1} & #2 \\
      \textit{\small#3} & \textit{\small #4} \\
    \end{tabular*}\vspace{-7pt}
}

\newcommand{\resumeSubSubheading}[2]{
    \item
    \begin{tabular*}{0.97\textwidth}{l@{\extracolsep{\fill}}r}
      \textit{\small#1} & \textit{\small #2} \\
    \end{tabular*}\vspace{-7pt}
}

\newcommand{\resumeProjectHeading}[2]{
    \item
    \begin{tabular*}{0.97\textwidth}{l@{\extracolsep{\fill}}r}
      \small#1 & #2 \\
    \end{tabular*}\vspace{-7pt}
}

\newcommand{\resumeSubItem}[1]{\resumeItem{#1}\vspace{-4pt}}

\renewcommand\labelitemii{$\vcenter{\hbox{\tiny$\bullet$}}$}

\newcommand{\resumeSubHeadingListStart}{\begin{itemize}[leftmargin=0.15in, label={}]}
\newcommand{\resumeSubHeadingListEnd}{\end{itemize}}
\newcommand{\resumeItemListStart}{\begin{itemize}}
\newcommand{\resumeItemListEnd}{\end{itemize}\vspace{-5pt}}

\definecolor{Black}{RGB}{0, 0, 0}
\newcommand{\seticon}[1]{\textcolor{Black}{\csname #1\endcsname}}


%-------------------------------------------%
%%%%%%  RESUME STARTS HERE  %%%%%

%----------HEADING----------%
%----------HEADING----------%
\begin{center}
	\textbf{\Huge \scshape Li-Yin Young} \\ \vspace{5pt}
	\small 720-454-9625 \quad
	{\texttt{|} \quad Superior, CO} \quad
	\href{mailto:lilyyoung1122@gmail.com}{\texttt{|} \quad \underline{lilyyoung1122@gmail.com}} \quad
	
\end{center}

%-----------EDUCATION-----------%
\section{Education}

\resumeSubHeadingListStart

    \resumeSubheading
    {University of Colorado Boulder}{2018-2020}
    {Master of Science in Applied Mathematics} {Boulder, CO}
  
     \resumeSubheading
    {University of Colorado Boulder}{2013-2015}
    {Master of Science in Computer Science}{Boulder, CO}
    
\resumeSubHeadingListEnd

%-----------EXPERIENCE-----------%
\section{Experience}

\resumeSubHeadingListStart

\resumeSubheading
{Cooperative Institute for Research in Environmental Sciences}{Boulder, CO}
{Software Engineer}{April 2021 - Present}

\resumeItemListStart
   	 \resumeItem{Supervising three software developers and mentoring them to become proficient in Git, unit testing, and server and network management.}
   \resumeItem{Led the engineer team to develop Python libraries for geomagnetic models. Each magnetic model shared the same upstream libraries but has its own API  and Git version control.}
    \resumeItem{Managed two open-source Git repositories. Responsible for verifying pull requests from external users and deciding whether to merge them.}

 
  	 \resumeItem{Led the software development for the NCEI’s geomagnetic models: HDGM. }
  \begin{itemize}
  	\item Steadily led my team in releasing the HDGM software and web API annually since joining CIRES.
  	\item Developed and deployed the magnetic model to the backend of web API using C and Java Servlet. 
  	\item Maintain the cloud-based backend  and includes APIs for high-resolution models.
  	\item Developed and deployed a logging tool to track user requests, user IPs, and server responses for the HDGM web API by Mockito and Java Serlvet.
  	\item Update the dependency of machine learning backend from Python2 to Python3 with unittest and replacing LSTM layer with GRU with tensorflow. Deploy the machine learning model to the GCP based web service.
  	
  \end{itemize}
  
   \resumeItem{Led the software development for the NCEI’s geomagnetic models: WMM online calculator, WMM Python API, WMM C softwares and WMM GUI.}
  \begin{itemize}
  	\item Deployed the WMM high-resolution model to the current backend of the WMM online calculator, WMM C software, and WMM GUI by C.
  	\item Collaborated with the NCEI IT team to develop the CI/CD process for the WMM online calculator, enabling deployment of the model backend, web API backend, and frontend to production.
  	\item Integrated components from data preprocessing to the model into the Python API and deployed it across development, testing, and production tiers.
  	
  \end{itemize}
  
   \resumeItem{Updated the shared library for the software backend of geomagnetic models, such as HDGM and WMM, to comply with the requirements of the NCEI IT security review.}
  
\resumeItemListEnd

\resumeSubheading
{Main Street Exchange}{Boulder, CO}
{Full Stack Developer}{Jun 2016 - May 2016}

\resumeItemListStart
    \resumeItem{Implemented scripting tools and virtual server environments to troubleshoot real-time system issues.}
    \resumeItem{Developed major functionality on the website's portal including third-party app integration and database management.}
\resumeItemListEnd

\resumeSubheading
{Topic Technology} {Boulder, CO}
{Machine Learning Developer} {Jan 2016 - May 2016}
\resumeItemListStart
    \resumeItem{Built software tools for extracting unstructured sentimental information from social media for training machine learning model.}
\resumeItemListEnd

\resumeSubheading
{Millennium Venture Systems} {Colorado Spring, CO}
{Machine Learning Engineer Internship}{Jul 2014 - Aug 2014}

\resumeItemListStart
    \resumeItem{Built the support vector machine(svm) application on time series prediction with C++.}
\resumeItemListEnd

\resumeSubHeadingListEnd


%-----------SKILLS-----------%
\section{Skills}
 
\begin{itemize}[leftmargin=0.15in, label={}]
	\small{\item{{}
	
        \textbf{Programming Languages}{: Python, C, Java, MySQL, bash, Linux CLI} \\
		\textbf{Development Frameworks}{: Tensorflow, Docker, GCP, Git}\\
}}
\end{itemize}



%-----------Publications----------%
\section{Research Publications}

\nocite{*}
\printbibliography[heading={subbibliography},title={Journal Publications}, type=article]
\printbibliography[heading={subbibliography}, title={Conferences}, type=inproceedings]




%-------------------------------------------%
\end{document} 
